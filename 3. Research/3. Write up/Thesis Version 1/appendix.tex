%*****************************************************
%	APPENDIX
%*****************************************************
\appendix

\chapter{Numerical modeling}

\section{Modeling of polar stochastic processes}
\label{app:modelling}
In this section, modeling techniques for the polar region are explored. Here, the focus is given on developing models for Polar sea ice mechanics and dynamics. An overview of these models is given along with a description of the variables as well as the scope of each model.

\subsection{Numerical Modeling of Sea Ice}

The Hibler model is a numerical designed to investigate sea ice dynamics and thermodynamics in the Arctic region \cite{hibler1979dynamic}. This model attempts to couple the sea ice dynamics to Sea ice thickness and uses this relationship to investigate the relationship between the effects of sea ice and the climate. Work so far has largely studied these effects independently using factors that largely ignore the inherent mechanical properties of Sea Ice \cite{hibler1979dynamic}. Coupling these effects would allow for a more general descriptor of Sea Ice spread regions.\par

The model is based off \textcite{coon1974modeling} AIDJEX \cite{hibler1979dynamic}, who use plastic-elastic constitutive laws to describe large-scale sea Ice spreads. It is assumed that cracks, ridges, and leads are randomly distributed on large scales \footnote{100 km from \textcite{coon2007arctic}}. While the Hibler model is not as complex, it is more robust as it allows for larger time-steps and simplifies system boundaries. Here, sea ice is modelled using similar viscous plastic laws \cite{hibler1979dynamic} that allow for non-linear plastic flows to be modelled without severe limitations by large time-steps. The model uses the following components:

\begin{enumerate}
	\item    Momentum balance  - air and water stress
	\item    Coriolis force
	\item    Inertial forces
	\item    Constitutive laws - ice stress, strain, strength
	\item    Ice thickness distribution - accounting for open water patches, changes in thickness and Concentration
	\item    Ice strength 
\end{enumerate}
\begin{equation}
	\frac{mDu}{Dt} = -mfk\times u +\tau_a +\tau_w -mg \nabla H +F 
\end{equation}

$\frac{D}{Dt}$ is the substantial time derivative, k is a unit vector, u is the sea ice velocity, m is the ice mass and f is the Coriolis parameter. Forces in the equation $\tau_a$, $\tau_w$ represent the stress of the air and water respectively where F is the force related to the internal ice stresses. H is the sea surface dynamic height and g is the acceleration due to gravity. Assuming constant turning angles, The air and water momentum equations are as follows
\begin{equation}
	\tau_a = \rho_a C_a|U_g|(U_g cos(\phi)+k\times U_g sin(\phi))
\end{equation}

\begin{equation}
	\tau_w = \rho_w C_w|U_w-u|[(U_w-u)cos(\phi)+k\times(U_w-u)sin(\phi)]
\end{equation}


where $\rho_a $ and $\rho_w$ are the densities of air and water, $C_a$/$C_w$ are the drag coefficients, $U_g$ is the geostrophic wind and $U_w$ is the geostrophic ocean current\par

The Hibler model is the de facto numerical model for large scale ice process \cite{Rutgher2019SmallScale}. The model is used to describe an area of 10 - 100km$^2$, Small scale models are still in development \cite{Rutgher2019SmallScale}.\par 

\subsubsection{Numerical Modeling of Ocean Waves}
Ocean waves are comprised of multiple spectral components with different magnitudes and wave periods knowledge of these spectral components is important for understanding the wave attenuation model \cite{williams2013wave} where, assuming the ice is modelled as a viscous fluid, wave energy is exponentially attenuated \cite{meylan2014situ}\cite{williams2013wave} with distance travelled into the ice due to partial reflections with the ice floes. The rate of attenuation is dependant on the wavelength however an exact mathematical relationship has not been found. The major issue with verifying these models is the lack of robust data availability \cite{meylan2014situ} thereby reaffirming the need for in-situ measurements.\par 
%%% TODO: INCLUDE WILLIAMS WAVE MODELS AND EQUATIONS FOR WAVE SPECTRUM ANALYSIS
\textcite{williams2013wave} describe three fundamental components of Waves in Ice Modeling. These are advection, attenuation, and ice breakage \cite{williams2013wave}. Advection and Attenuation describe how energy transfer occurs between waves and ice and are dependant on the group velocity $c_g$ and the attenuation factor $\hat{\alpha}$ which, in turn, are dependant on the frequency of the wave \cite{williams2013wave}. Also, the properties of ice are significant. These include Young's modulus $Y$, Poisson Ratio $\nu$, strain $\epsilon$ and viscous damping parameter $\Gamma$. The initial Floe Size Distribution and sea ice concentration are also considered. The assumption is that wave breakage feeds back into the model with a new Floe Size distribution \cite{williams2013wave}. \par

Wave advection is described by the following energy model:
\begin{equation}
	\frac{1}{c_g}(\partial_t +c_g\partial_x)S(\omega;x,t) = R_{in}- R_{ice} - R_{other}- R_{nl}
\end{equation}
where $R_{in}$ is the wind input energy, $R_{ice}$,$R_{nl}$,$R_{other}$ represent the energy loss from ice, other sources as well as non linear energy exchanges. $S(\omega;x,t)$ represents the waves in terms of its energy spectral density \cite{williams2013wave} For this model, the energy input is considered to come only from the Rate of exchange between ocean and Ice. Hence all other energy rates are considered 0 and $R_{ice}$ is defined in terms of $\hat{alpha} \text{ and } S$
\begin{equation}
	\frac{1}{c_g}(\partial_t +c_g\partial_x)S(\omega;x,t)  = -\hat{\alpha}(\omega,c,h,\langle D \rangle)S(\omega;x,t)
\end{equation}

$\hat{\alpha} = \frac{\alpha}{\langle D \rangle}$ describes the average attenuation per ice floe. In terms of Ice thickness and wave period \cite{williams2013wave}. By this definition, $R_ice$ is quasi linear \cite{williams2013wave} since a wave with a significantly large Energy spectral density can break the floe decreasing the dimensions $\langle D \rangle$ and increase the dimensional attenuation factor $\hat{\alpha}$. The  operator $(\partial_t +c_g\partial_x)$ serves as the lagrangian reference fram at a moving velocity $c_g$. Finally, by breaking the above model into:
\begin{subequations}
	\begin{align}
		\frac{dx}{dt} = c_g(\omega,t_*,x) \label{advect}\\
		\frac{dS(\omega;x,t)}{dx} = -\hat{\alpha}(\omega,x,t_*,S_*)S(\omega;x,t) \label{atten}
	\end{align}
\end{subequations}

we can describe the dynamics of the sea ice during a breaking event at a time $t_*$ \cite{williams2013wave}. Hence, the model is broken up into an advection model in \ref{advect} and an attenuation model in \ref{atten}.\par

The next step in the model is determining the mathematical model for wave energy. A stochastic approach is taken to define key wave parameters \cite{williams2013wave}. The Significant wave height is found using the formula
\begin{equation}
	H_s = 4\sqrt{m_0[n]}
\end{equation}
$m_n[\eta]$ describes the mean square surface sea elevation of a particle and is derived from the Spectral Density $S$ \cite{williams2013wave}.
\begin{equation}
	m_n[\eta] = \int_{0}{\infty}S(\omega)\omega^nd\omega
\end{equation}

The significant wave height can be considered 4 times the standard deviation of the surface elevation \cite{meylan2014situ}. finally, by determining the significant wave height, the dominant wave period can be calculated as $\frac{1}{f_d}$ where $f_d$ is the frequency at which the dominant wave period occurs \cite{meylan2014situ}.

\section{Modeling of GPS dilation of precision}
\label{appendix:GPS_DOP}
 Given a user's position on the earth, the distance from the user to the satellite is characterised by the equation:
\begin{equation}
	r =  s - u
\end{equation}
where $r$ is the distance from the user to the satellite, $s$ is the distance from the earth's centre to the satellite and u is the distance from the earth to the user. By measuring the propagation time from the user to the satellite, The absolute distance $||r||$ can be calculated and hence, the pseudo-range can be calculated as
\begin{equation}
	\rho_i = ||s_i-u||+ct_b + v_{\rho_i}
\end{equation}
where $\rho_i$ is the pseudorange for satelite i, c is the speed of light, $t_b$ is  the clock offset and $v_{\rho_i}$ is the noise of the pseudorange measurement and:
\begin{equation}
	||s_i-u|| = \sqrt{(x_i - x_u)^2+(y_i-y_u)^2+(z_i-z_u)^2} \text{ for } i \in 1,2,3...N \label{los}
\end{equation}
where $N$ is the number of satellites and $(x_i,y_i,z_i)$ is the 3 dimensional position of satellite $i$. This represents a non-linear relationship for the line of sight of a satellite.  \textcite{jwo2001efficient} explains that by creating a Taylor series centered on a nominal user position $(\hat{x_n},\hat{y_n},\hat{z_n})$ and ignoring the higher terms \cite{jwo2001efficient}. It then follows that:
\begin{equation}
	\Delta\rho_i = \rho_i - \hat{\rho_i} = e_{i1}\Delta x_u + e_{i2}\Delta x_u +  e_{i3}\Delta z_u
\end{equation}

The terms $e_{ij}$ represent the line of sight vector $E_i$ whereas the term $\hat{\rho_i}$ is the pseudo-range at the nominal user's position. It follows that the vector $E_i$ can be calculated as follows \cite{jwo2001efficient}.
\begin{subequations}
	\begin{align}
		e_{i1} = \frac{\hat{x_n} - x_i}{\hat{r_i}}\\
		e_{i2} = \frac{\hat{y_n} - y_i}{\hat{r_i}}\\
		e_{i3} = \frac{\hat{z_n} - z_i}{\hat{r_i}}\\
		\hat{r_i} = \sqrt{(\hat{x_n} - x_i)^2+(\hat{y_n} -y_i)^2+(\hat{z_n} -z_i)^2}
	\end{align}
\end{subequations}

Given $n$ satellites, the equation \eqref{los} can be written as a matrix with the following form:
\begin{equation}
	\textbf{z} = \textbf{Hx}+ \textbf{v}
\end{equation}
\begin{equation}
	\Delta \rho _i = \begin{bmatrix}
		\Delta \rho_1 &  \Delta \rho_2 &  \Delta \rho_3 & ... & \Delta \rho_n
	\end{bmatrix} 
\end{equation}
where 
\begin{subequations}
	\begin{align}
		\textbf{H} = \begin{bmatrix}
			e_{11} & e_{12} & e_{13}& 1 \\
			e_{21} & e_{22} & e_{23}& 1 
			\\
			e_{31} & e_{32} & e_{33}& 1
			\\
			... & ... & ... &  1 
			\\
			e_{n1} & e_{n2} & e_{n3} & 1
		\end{bmatrix}\\
		\textbf{x} = \begin{bmatrix}
			\Delta x_u \\ \Delta y_u \\\Delta z_u \\ c\Delta t_b
		\end{bmatrix}\\
		\textbf{v} = \begin{bmatrix}
			v_{\rho_1}\\
			v_{\rho_2}\\
			v_{\rho_3}\\
			...\\
			v_{\rho_n}
		\end{bmatrix}
	\end{align}
\end{subequations}

The Matrix \textbf{H} is $n\times4$ where $n \geq 4$ to calculate all the parameters for GDOP \cite{jwo2001efficient}. We can then solve for the vector \textbf{x} by taking the psuedo inverse of H i.e $\hat{\textbf{x}} = (\textbf{H}^T\textbf{H})^{-1}\textbf{H}^t\textbf{z}$. Hence, given that the psuedo range is linearised, the quality of navigation is taken as the difference between the estimated position and the actual position \cite{jwo2001efficient}.
\begin{equation}
	\Tilde{\textbf{x}} = \hat{\textbf{x}} - x = (\textbf{H}^T\textbf{H})^{-1}\textbf{H}^Tv
\end{equation}
$E\{\Tilde{\textbf{x}}\Tilde{\textbf{x}}^T\}$ describes the covariance between the errors in the components of the estimated position \cite{jwo2001efficient} and is calculated as 
\begin{equation}
	E\{\Tilde{\textbf{x}}\Tilde{\textbf{x}}^T\} = (\textbf{H}^T\textbf{H})^{-1}\textbf{H}^TE\{\textbf{vv}^T\} (\textbf{H}^T\textbf{H})^{-1}\textbf{H}
\end{equation}
where $E\{\textbf{vv}^T\} = \sigma^2 I$. If all components of $\sigma$ are uncorrelated then the covariance becomes 
\begin{equation}
	E\{\Tilde{\textbf{x}}\Tilde{\textbf{x}}^T\} = \sigma^2(\textbf{H}^T\textbf{H})^{-1}
\end{equation}
and thus the GDOP factor can be calculated from the RMS values of $\sigma^2$ i.e.
\begin{equation}
	GDOP = \frac{\sqrt{\sigma_{xx}^2+\sigma_{yy}^2+ \sigma_{zz}^2+\sigma_{tt}^2}}{\sigma}
\end{equation} where $\sigma_{xx}^2$,$\sigma_{yy}^2$,$\sigma_{zz}^2$,$\sigma_{tt}^2$ are the RMS values of the x,y,z time components respectively. THe value GDOP can also be decomposed into the positional dilation of precision (PDOP), time dilation of precision (TDOP), horizontal dilation of precision HDOP, and vertical dilation of precision (VDOP) which characterise the effects satellite spread on the 3 dimensional position, time, horizontal position and altitude respectively.\par

\section{Numerical Techniques for modeling ocean waves}

\subsection{Kuik Method}
\label{kuik}
The Kuik method, developed by \textcite{kuik1988method} is a computational technique for measuring and determining the directional characteristics of ocean waves. Measurement of these characteristics are derived from the pitch and roll of an ocean buoy are measured. By using an accelerometer, gyroscope or an inertial measurement system to measure the slope and heave of the 3 axes \cite{kuik1988method}, it is possible to reconstruct the sea state given a set of data provided the data is of a specific length sampled above the Nyquist frequency of dominant ocean swells. A major advantage of the Kuik method is that the parameters are estimated directly from the Fourier transform of the measured signal \cite{kuik1988method} without assumptions about the model. Should the algorithm be used to measure waves in ice, no information is required about the dynamics of the model of the ice floe. This greatly improves the accuracy since ice floes can vary in width, distribution and area as well as change shape due to collisions, freezing and melting. Wind waves are described using a two-dimensional energy spectrum $E$ with wave energy spread over a frequency $f$. The normalised distribution of energy over direction is defined according to \textcite{kuik1988method} as
\begin{equation}
	D_f(\theta) = \frac{E(f,\theta)}{\int_0^{2
			\pi}E(f,\theta)d\theta}
\end{equation} 

Finally, by computing the model per frequency, the distribution simplifies to $D(\theta)$ which can be approximated by a Fourier series with 4 terms \cite{kuik1988method} derived from the pitch, roll and heave of a buoy. Finally, the model can fully characterise the wave spectrum by calculating the following parameters from the Fourier coefficients:

\begin{enumerate}
	\item mean wave direction $\theta_0$
	\item directional width $\sigma$
	\item skewness $\gamma$
	\item kurtosis $\delta$ 
\end{enumerate}

The accuracy of the mean wave direction and width is affected by noise in the sampled data. small RMS values of noise can result in rapid increases of directional width by 1\% to 5\% \cite{kuik1988method}. Additionally, pitch and roll buoys are not free particles. They have an associated mass and therefore an associated inertia \cite{kuik1988method}. This results in a phase shift of the Fourier term by $\phi_i i \in {x,y,z}$  in the first harmonic \cite{kuik1988method}. This shift can result in an error of $0.5^\circ $ for $\sigma > 25 ^\circ $ to $1^\circ \text{ for } \sigma < 10^\circ $ \cite{kuik1988method}   

\subsection{Welch-Earle Method}
\label{welchearl}
The Welch-Earle Method is an algorithm for calculating either directional and non-directional wave data depending on the assumptions of the input data \cite{earle1996nondirectional}. Data is derived from the vertical acceleration, roll and pitch of a buoy oriented perpendicular to the surface on either a vertically stabilised platform or a hull-fixed platform \cite{earle1996nondirectional}. Directional wave data is determined from both the acceleration, roll, pitch and heave from the buoy while non-directional wave data is calculated from time-series acceleration only. In this method, a digital time series representation of the vertical Acceleration along with 2 orthogonal Gyroscope measurements and Magnetometer readings relative to the earth’s magnetic field is obtained. The method accounts for the response function of the buoy and provides corrections to phase differences as a result of the buoy's inertia \cite{earle1996nondirectional}. Full directional and non-directional wave data is characterised by calculating the spectra and co-spectra of the time series data. The first part of the method is developed by \textcite{welch1967use} and is used to calculate the power spectral density.
Given a discrete time series data X(j) with a power spectral density P(f), |f|< $\frac{1}{2}$.this data segmented into a set of k-bins $X_k(j) | j \in {0,L-1}$ \cite{welch1967use}. Each bin is multiplied by a selected window function W(k) of length L. Additionally, Bins are taken with a 50\% overlap to produce better statistical averages \cite{earle1996nondirectional} The Fast Fourier Transform (FFT) of the result is taken to for a periodograms $I_k$ \cite{earle1996nondirectional}. Finally, the new power spectral estimator $\hat{P}(f_n)$ is calculated by taking the average of the K periodograms as shown in \textcite{welch1967use}

\begin{equation}
	\hat{P}(f_n) = \frac{1}{K}\sum^K_{k=1}I_k(f_n)
\end{equation}

where 
\begin{equation}
	f_n = \frac{n}{L} | n \in 0,1,2... \frac{L}{2}
\end{equation}

Detrending is used to account for the effects of buoy motion on the time series. The inertia of the platform results in non-zero mean trends often as a result of constant wind or currents acting on the hull of the buoy \cite{earle1996nondirectional}. These must be discarded before the spectrum and co spectra are calculated. The resolution of the accelerometer is important for accurately tracking acceleration \cite{kohout2015device}. If the resolution is too small, low accelerations will not be recorded resulting in incorrect vertical accelerations being calculated. \textcite{kohout2015device} found that a low-resolution IMU was unable to reliably flag that it had exceeded a boundary condition and hence it was discarded \cite{kohout2015device}\par 

The spectra and co-spectra of the directional and non-directional wave series can be calculated by computing the spectrum $S(x)$ as a function of frequency and direction (as is similar to the Kuik method). \textcite{earle1996nondirectional} show that characterisation of the co-spectra $C(x)$ and spectrum$S(x)$ can be achieved by calculating the first 4 Fourier coefficients. Finally, the sea state can be represented by calculating the following parameters

\begin{enumerate}
	\item Longuet-Higgins directional parameters
	\begin{enumerate}
		\item $a_0$
		\item $a_1$
		\item $b_0$
		\item $b_1$
	\end{enumerate}
	\item Significant Wave Height $H_0$
	\item Dominant Wave Period $T_p$
	\item Total Degrees of Freedom $TDF$
	\item Average zero-crossing period $T_{av}$
	\item Zero-crossing period $T_{zero}$
\end{enumerate}

This approach brings into account the possibility of spectral leakage however, this can be greatly minimised by sampling above the Nyquist frequency of the upper Wave frequency band (generally taken to be 0.5Hz) for a minimum of 1000 seconds (about 16 – 17 minutes). Additionally, spectral leakage can be reduced by selecting a window function with a gradual taper such as a half cosine or Hanning taper \cite{welch1967use}\par

\section{Temperature sensing measurement techniques}
\label{appendix:tempsense}
\subsubsection{Thermistors}

Modern thermistors have progressed significantly in the past decade. Up until recently, they have been considered inaccurate with uncertainty ranges of up to 5\% \cite{tong2001improving}. Thermistors are capable of providing accuracies of up to 0.01$^\circ$. They consist of a semiconductor that changes its resistance in response to temperature \cite{childs2000review}. They have a faster response time than RTDs and work on the same principle for temperature measurement. However, where RTDs have a Positive temperature coefficient, thermistors have a negative temperature coefficient \cite{tong2001improving}. These devices can operate over a substantial, albeit relatively limited, range of $-100^\circ - 300^\circ C$. The major trade-off with these devices is the lack of standards \cite{tong2001improving}. Operating the device involves a large degree of uncertainty. Also, these devices are not powerful enough to accurately reach the desired ranges alone. They need to be coupled with similar devices. Finally, the response curve is non-linear. The  relationship between resistance and temperature is \cite{childs2000review}:
\begin{equation}
	R_T = R_0e^{1 - B(\frac{1}{T}- \frac{1}{T_0})}
\end{equation}
where $R_T$ is the temperature measured, $R_0$ is the resistance at a known temperature $T_0$, T is the temperature and B is a coefficient based on the properties of the thermistor. Finally, these devices are more prone to noise from excitation current.

\subsubsection{Silicon Temperature Devices}

Semiconductor temperature devices are suited to applications where the temperature ranges from -55 to 150 $^\circ C$ these devices are capable of providing a stable output with a typical accuracy of $0.8 ^\circ C$. These devices typically consist of diodes and transistors with a band gap voltage that changes with a change in temperature \cite{childs2000review}. These devices are advantageous in electronic application due to their small form, high accuracy and stability. These devices are relatively simple and have a good sensitivity to changes \cite{childs2000review}. Diodes are typically used in semiconductor devices. Here, the forward voltage drop across the p-n junction is linearly proportional to the Ambient temperature over a certain temperature range (typically 25K - 400K) \cite{childs2000review}. These devices are made out of either silicon or Gallium-Arsenide. Silicon is preferred as it has better stability at low temperatures and is cheaper however, this comes at the trade-off of a lower voltage output \cite{childs2000review}. \par These Types of devices are readily available in IC forms and are manufactured in a variety of packages, types and compositions for any application. Typical devices are DS18B20, LM355 or BMP2080. Recent innovations in Silicon sensing have seen the rise of CMOS devices and Micro Electrical-Mechanical Systems (MEMS) being used more frequently \cite{mansoor2015silicon}. While these devices can suffer from deterioration due to self-heating, \textcite{mansoor2015silicon} discuss that the low-power operation of these devices can offset this issue. This is advantageous for systems that are constrained by power consumption. However, a major disadvantage with these devices is that these devices work ideally with a purely DC signal.An AC coupled signal can cause significant errors in the output \cite{childs2000review} \cite{mansoor2015silicon}. These errors can be the result of improper shielding and poor grounding. Hence proper shielding and grounding are required to reduce these errors. Finally, these devices require careful calibration before use.

\section{Pressure sensing techniques}

In this section, common pressure sensing techniques are discussed as well as their applications in remote sensing. Additionally, the numerical model of each sensor is given showing the relationship between the measured variable and the output signal. 

\subsubsection{Diaphragm based sensors}

The current state of pressure sensing technology is driven towards Miniature MEMs version of large scale devices \cite{eaton1997micromachined}. Most large scale pressure sensors consist of a diaphragm mounted on a device in a known way. The diaphragm is coupled to a device that converts the pressure to a mechanical movement which is then measured using a gauge. These senses often had a secondary sensor that would convert the mechanical movement to an electrical signal which was then measured \cite{eaton1997micromachined} these sensors determine pressure by measuring the deflection of a miniature diaphragm. This deflection is converted to an electrical signal. Typically, a reference pressure is provided as a measurement of a sealed chamber or absolute pressure port. Assuming the simplest version of this sensor i.e. a plate of uniform thickness \cite{eaton1997micromachined} The deflection $w$ of the diaphragm is related to the pressure  $P$ by the following equation: \cite{eaton1997micromachined}

\begin{equation}
	w(r) = \frac{Pa^4}{64D}(1- (\frac{r}{a})^2)^2
\end{equation}

where r is the deformed radius of the diaphragm, a is the original radius and D is the  rigidity of the diaphragm governed by the equation:

\begin{equation}
	D = \frac{Eh^3}{12(1-v^2)}
\end{equation}
where E,h,v are Young's modulus, thickness and Poisson's ratio of the disc \cite{eaton1997micromachined}. This technique suffers from a multitude of problems namely, the diaphragm is susceptible to plastic deformation and more robust diaphragms result in more complex relationships. The current relationship is nonlinear and can result in calculation errors. Eaten (1997) advocate for the use of MEMs based electronics on these principles.

\subsubsection{Piezoresistive sensors}

Piezoresistive sensors are electric devices constructed out of a semiconductor whose electrical properties change when a stress is applied \cite{eaton1997micromachined}. these devices are mounted to a diaphragm and exhibit a linear change in resistance with a change in Pressure. Currently, these sensors take the form of single-crystal diaphragms with piezoelectric resistors diffused through the materials. The advantage of these devices is robustness towards hysteresis and measurement drift. At low temperatures, silicon exhibits near-perfect elastic behaviour and is  3 times the tensile strength of strain gauges\cite{eaton1997micromachined}. The sensors are, however susceptible to thermal expansion and can exhibit significant temperature drift \cite{samaun1971ic}. Additionally, these sensors require resistors with identical temperature Resistance characteristics otherwise the measurements will be inaccurate. Finally, additional compensation techniques are required.

\subsubsection{Capacitive sensors}

These sensors consist of parallel conductive plates. Assuming a constant, known dielectric, an external pressure causes the plates to deform which changes the capacitance C according to the relationship \cite{eaton1997micromachined}
\begin{equation}
	C = \int \int \frac{\epsilon}{d - w(r)}drd\theta
\end{equation}

where $w(r)$ is the deformation of the plate, $\epsilon$ is the strain experienced on the plate and d is the distance of separation. The Pressure capacitance relationship can be approximated using a least-squares fit \cite{eaton1997micromachined} however this results in model errors of 1.5\% and up to 11\% at $w = \frac{1}{2}h$ the height of the plate. These sensors are more advantageous over piezoresistive sensors as they have higher pressure sensitivity and reduced susceptibility to temperature drift. However, these sensors are significantly susceptible to parasitic capacitance which can result in losses and errors. Additionally, these sensors are simple in design however they tend towards more complex circuit requirements.

\newpage
\chapter{Project Design}

\section{Stakeholder analysis}
\label{app:stakeholder}
For this project, a stakeholder is defined as any user or set of users who will impact the overall project or be impacted by the final design of the project \cite{varvasovszky2000stakeholder}. Therefore, the stakeholders for this project can be considered as any individual directly involved in the designing/building of the project or a user: i.e. an individual responsible for using the system or any data it generates. The approach was to first identify the key stakeholders of the project and identify their role and involvement. Information was obtained from face to face meetings with the stakeholders which was used to generate the table shown below.

\begin{table}[H]
	\centering
	\caption{Table showing the key stakeholders in the project, their level of involvement as well as their interests in the project}
	\label{tab:stake}
	\setlength{\extrarowheight}{5pt}
	\resizebox{\textwidth}{!}{%
	\begin{tabular}{ | >{\RaggedRight}m{0.3\textwidth} | >{\RaggedRight}m{0.4\textwidth} |>{\RaggedRight}m{0.4\textwidth} | }
		\hline
		Stakeholder &  Function & Involvement\\
		\hline
		Lead Scientist \hfill & Principal stakeholder: Initiates and funds the SHARC buoy project & Sets the user requirements, provides feedback on progress, organises expeditions.  \\
		\hline
		Project Supervisor & Set the project timeline and provide feedback on progress.\hfill& Primary engagement with principle Stakeholder\\
		\hline
		Project Engineer & Translate specifications to subsystems.& Selecting hardware, sourcing materials as well as designing firmware for the buoy. \\
		\hline
		Deployment Team & Place the system in a selected location and ensure the device is communicating & Physical handling of the device, understanding how the system works.\\
		\hline
		User & Collect and archive data packets from the buoy & Interact with the data portal and decompression software.\\
		\hline
		Collaborators & Work with users on interpreting data from the system & Analysing generated data sets.\\
		\hline
	\end{tabular}}
\end{table}

\section{Acceptance test protocols}
\label{app:atp}


\begin{table}[H]
	\centering
	\caption{Acceptance test for subsystem connectivity testing}
	\setlength{\extrarowheight}{5pt}
	\resizebox{\textwidth}{!}{%
	\begin{tabular}{|m{0.25\textwidth}|m{0.75\textwidth}|}
		\hline
		
		\textbf{AT001 }& \textbf{Connection test} \\
		\hline
		\textbf{Evaluation type:} & Software unit test\\
		\hline
		\textbf{Target: } & Sensors \\
		\hline
		\textbf{Test protocol:} & microcontroller should be connected to the device on a specified communication port.The microcontroller then requests an acknowledgment from the device either by reading the id register or by asking the device to return an acknowledgment string.\\
		\hline
		\textbf{Pass Condition} & \vspace{5pt} \begin{itemize}
			\item microontroller receives acknowledgment
			\item ID register read and valid 
		\end{itemize} \\
		\hline
		\textbf{Fail Condition:} & \begin{itemize}
			\item Incorrect ID register value returned
			\item NACK Returned 
			\item Invalid message string (timing error or framing error)
			\item No data received (receiver timeout - malfunctioning device)
			\item Failure to request read (transmission timeout - No device available)
		\end{itemize}\\
		\hline
	\end{tabular}}
	
	\label{tab:AT001}
\end{table}

\begin{table}[H]
	\centering
	\caption{Acceptance test for fault testing.}
		\setlength{\extrarowheight}{5pt}
	\resizebox{\textwidth}{!}{%
	\begin{tabular}{|m{0.25\textwidth}|m{0.75\textwidth}|}
		\hline
		
		\textbf{AT002 }& \textbf{Fault testing} \\
		\hline
		\textbf{Evaluation Type:} & System Recovery\\
		\hline
		\textbf{Target: } & Hardware subsystems\\
		\hline
		\textbf{Test Protocol:} & Connect Subsystem to a micro-controller and run Acceptance Test AT001 under the following circumstances
		\begin{itemize}
			\item \textbf{Nack Test:} Change Acknowledgement string (USART Peripheral) or device ID (SPI/I2C) to an incorrect value.
			\item \textbf{Corrupted Message Response Test:} modify the baud rate to produce a corrupted message.
			\item \textbf{Disconnection Test:} Set the system to run a continuous cycle. Unplug the device while the system is running.
			
		\end{itemize} Evaluate return status. \\
		\hline
		\multirow{3}{*}{\textbf{Expected Response}} & \textbf{Nack Test: }  Controlled Exit and return "NACK\_ERROR". System clears message buffer and retries.\\ 
		\cline{2-2}
		& \textbf{Corrupted Message: } Controlled Exit return "MESSAGE\_ERROR". System re-initialises communication peripheral with different baud rate and retries.\\
		\cline{2-2}
		& \textbf{Disconnection Test: } Communication Timeout, controlled exit + return "TX\_ERROR". Critical Failure: system recognises that device is missing and continues routine without it.\\
		\hline
		\textbf{Pass Condition:} &\vspace{5pt} \begin{itemize}
			\item Device returns status and handles faults in a predicted manner  
			\item Critical Failures handled without errors
		\end{itemize} \\
		\hline
		\textbf{Fail Condition:} & \vspace{5pt} \begin{itemize}
			\item Device freezes during test
			\item Device returns incorrect status
			\item Segment Faults
			\item Hard faults
			\item Software Rest occurs during test
		\end{itemize}\\
		\hline
	\end{tabular}}
	
	\label{tab:AT002}
\end{table}

\begin{table}[H]
	\centering
	\caption{Acceptance Test for component selection}
	\begin{tabular}{|m{0.25\textwidth}|m{0.75\textwidth}|}
		\hline
		\textbf{AT003 }& \textbf{Specification Test} \\
		\hline
		\textbf{Evaluation Type:} & Analytical \\
		\hline
		\textbf{Target: } & Hardware components\\
		\hline
		\textbf{Test Protocol:} & Evaluate Specifications of the components from the data sheet to determine if the specifications meet the requirements for the system \\
		\hline
		\textbf{Pass Condition} & \vspace{5pt} \begin{itemize}
			\item Specifications meet the desired requirements
		\end{itemize} \\
		\hline
		\textbf{Fail Condition:} & \vspace{5pt} \begin{itemize}
			\item Specifications do not meet the requirement
		\end{itemize}\\
		\hline
	\end{tabular}
	
	\label{tab:AT003}
\end{table}

\begin{table}[H]
	\centering
	\caption{Acceptance Test for Subsystem Robustness Testing}
	\begin{tabular}{|m{0.25\textwidth}|m{0.75\textwidth}|}
		\hline
		
		\textbf{AT004 }& \textbf{Subsystem Robustness Test} \\
		\hline
		\textbf{Evaluation Type:} & Software \\
		\hline
		\textbf{Target: } & System, subsystem \\
		\hline
		\textbf{Test Protocol:} & Connect Subsystem to microcontroller and run a preset routine covering the following cycle \begin{itemize}
			\item Intialisation
			\item function
			\item Deinitialisation
		\end{itemize} Run this cycle 100 times consecutively.\\
		\hline
		\textbf{Pass Condition} & \vspace{5pt} \begin{itemize}
			\item Microcontroller sccessfully completes consecutive cycles
		\end{itemize} \\
		\hline
		\textbf{Fail Condition:} & \vspace{5pt} \begin{itemize}
			\item  failure to complete more than 1 consecutive cycle
			\item failure to initialise (fails acceptance Test AT001)
			\item failure to correctly deinitialise after completing routine
			\item Subsystem does not restart the cycle when reset
		\end{itemize}\\
		\hline
	\end{tabular}
	
	\label{tab:AT004}
\end{table}

\begin{table}[H]
	\centering
	\caption{Acceptance Test for accelerated system testing}
	\begin{tabular}{|m{0.25\textwidth}|m{0.75\textwidth}|}
		\hline
		\textbf{AT005 }& \textbf{Accelerated System Test} \\
		\hline
		\textbf{Evaluation Type:} & Software \\
		\hline
		\textbf{Target: } & System \\
		\hline
		\textbf{Test Protocol:} &  System to run firmware with all sensors initialised. Routine loaded on system that cycles between all the sensors and communication modules turning them on and off then cycling through deep sleep mode. This occurs over a 1 hour period\\
		\hline
		\textbf{Pass Condition} & \vspace{5pt} \begin{itemize}
			\item System successfully cycles through sensors with no timing delays
			\item System completes an hour of accelerated testing with no intervention
			\item Power Reset do not cause the device to lock up or malfunction
		\end{itemize} \\
		\hline
		\textbf{Fail Condition:} & \vspace{5pt} \begin{itemize}
			\item System freezes at any point during the test 
			\item System unable to turn on any sensro
			\item system unable to enter sleep mode
			\item system encounters unexpected reset
			\item system unable to run for an hour
		\end{itemize}\\
		\hline
	\end{tabular}
	
	\label{tab:AT005}
\end{table}

\begin{table}[H]
	\centering
	\caption{Acceptance Test for Subsystem Calibration Testing}
	\begin{tabular}{|m{0.25\textwidth}|m{0.75\textwidth}|}
		\hline
		
		\textbf{AT006 }& \textbf{Calibration Test} \\
		\hline
		\textbf{Evaluation Type:} & Statistical \\
		\hline
		\textbf{Target: } & Sensor Measurements \\
		\hline
		\textbf{Test Protocol:} & Connect Device to a data logger and set the measurand to a static value Record 100 sample points from the Device under test at a fixed frequency for a set amount of time. Measure against an accurate reference. Calculate mean and average value from data set and ensure it falls within the parameters given by the datasheet.Determine the disagreement between the average recorded value and average measured value and take the difference as the fixed offset bias. Repeat the Test twice more first by adjusting the value half through recording then by varying the value at a fixed rate \\
		\hline
		\textbf{Pass Condition} & \vspace{5pt} \begin{itemize}
			\item Calibration produces a consistent output well within the accepted error range when measured against a reference
			\item Step testing bring the measured value to the correct value
			\item device is capable of measuring over the specified range
		\end{itemize}\\
		\hline
		\textbf{Fail Condition:} & \vspace{5pt} \begin{itemize}
			\item  Test does not produce a predictable. consistent offset
			\item  Calibration values produces an invalid dataset
			\item Device under Test fails at any point.
			\item Calibrated Data set unable to replicate the measurand
		\end{itemize}\\
		\hline
	\end{tabular}
	
	\label{tab:AT006}
\end{table}

\begin{table}[H]
	\centering
	\caption{Acceptance Test for Power Test}
	\begin{tabular}{|m{0.25\textwidth}|m{0.75\textwidth}|}
		\hline
		\textbf{AT007 }& \textbf{Power System Test} \\
		\hline
		\textbf{Evaluation Type:} & Hardware\\
		\hline
		\textbf{Target: } & Power System \\
		\hline
		\textbf{Test Protocol:} & Connect the Power system to a Load of a known resistance. Connect an Ammeter and a Voltmeter respectively in series and in parallel. Measure the Current and Supply voltage at a fixed rate for 1 hour. Record the Battery Voltage before the Test and After the Test Then decrease the load to increase the current and Run until the Battery Voltage drops Below the Threshold for the Regulator. Measure the Output current and Supply Voltage. \\
		\hline
		\textbf{Pass Condition} & \vspace{5pt} \begin{itemize}
			\item Device cycles through routines for set period of time without failure
			\item Device survives for specified period of time
			\item Recorded values do not exceed the specs given from the data-sheet of the components
		\end{itemize} \\
		\hline
		\textbf{Fail Condition:} & \vspace{5pt} \begin{itemize}
			\item Power System depleted before test has finished
			\item Device fails to perform routine at any point during the test
			\item Mechanical/ Electrical failure occurs during test.
		\end{itemize}\\
		\hline
	\end{tabular}
	\label{tab:AT007}
\end{table}


\begin{table}[H]
	\centering
	\caption{Acceptance Test for Low Temperature Test}
	\begin{tabular}{|m{0.25\textwidth}|m{0.75\textwidth}|}
		\hline
		\textbf{AT008 }& \textbf{Low Temperature Tests} \\
		\hline
		\textbf{Evaluation Type:} & Hardware Robustness\\
		\hline
		\textbf{Target: } & Subsystem and full system \\
		\hline
		\textbf{Test Protocol:} & Connect subsystem to a datalogger and place in freezer. Set the freezer to $-20^\circ C$ and run the system through an accelerated subsystem test as per AT003. Then take the device out the freezer and wait for it to thaw. Then run another accelerated subsystem test. Finally connect all subsystems together and place in enclosure. Put device in the freezer and run an accelerated system Test as per AT004. Repeat in Room Temperature Conditions\\
		\hline
		\textbf{Pass Condition} & \vspace{5pt} \begin{itemize}
			\item System completes routine cycles in both sub zero and room temperature environment
			\item Subsystem Passes AT003 in $-20^\circ C$ and Room Temperature
		\end{itemize} \\
		\hline
		\textbf{Fail Condition:} & \vspace{5pt} \begin{itemize}
			\item Incorrect ID register value returned (SPI or I2C Address Incorrect)
			\item Subsystem Fails AT003 in sub-zero and room temperature environment available
			\item System fails AT004 in sub-zero and room temperature environment
		\end{itemize}\\
		\hline
	\end{tabular}
	\label{tab:AT008}
\end{table}

\begin{table}[H]
	\centering
	\caption{Acceptance Test for final system deployment test.}
	\begin{tabular}{|m{0.25\textwidth}|m{0.75\textwidth}|}
		\hline
		\textbf{AT009 }& \textbf{Deployment Test} \\
		\hline
		\textbf{Evaluation Type:} & Performance\\
		\hline
		\textbf{Target: } & Full System \\
		\hline
		\textbf{Test Protocol:} & Only to be performed once requirements for all other tests have been met. Ensure device is calibrated before hand and in a deploy-able state. Ensure power is turned on and sensors have been initialised. Deploy the system in a desired location and monitor the transmitted packages. \\
		\hline
		\textbf{Pass Condition} & \vspace{5pt} \begin{itemize}
			\item System acknowledges Deployment and routinely transmits full packets of data.
			\item System survives for 1 month or longer
		\end{itemize} \\
		\hline
		\textbf{Fail Condition:} & \vspace{5pt} \begin{itemize}
			\item No acknowledgement received
			\item No packets received
			\item Empty data received
		\end{itemize}
		Failure within the first half an hour of deployment should result in immediate retrieval of the device.\\
		\hline
	\end{tabular}
	\label{tab:AT009}
\end{table}


\section{Schematics and renders}
\label{app:appendix.schem}

\begin{figure}[H]
    \centering
\includegraphics[page= 1,width= 0.9\textwidth]{enclosure}
    \caption{Schematic of top enclosure which protects the electronics}
    \label{fig:top_schem}
\end{figure}

\begin{figure}[H]
    \centering
\includegraphics[page= 2,width= 0.9\textwidth]{enclosure}
    \caption{Schematic of bottom enclosure for the batteries and power system.}
    \label{fig:bot_schem}
\end{figure}

\begin{figure}[H]
    \centering
\includegraphics[page= 3,width= 0.9\textwidth]{enclosure}
    \caption{Schematic of the connector block which provides a base for the electronics to be mounted on. }
    \label{fig:conblock_schem}
\end{figure}

\begin{figure}[H]
    \centering
\includegraphics[page= 4,width= 0.9\textwidth]{enclosure}
    \caption{Full enclosure schematic.}
    \label{fig:full_schem}
\end{figure}

%Event/ Interrupt description
\chapter{Software design}

\section{Events and interrupt handling protocols} 
\label{sec:evt}

\begin{table}[H]
    \centering
    \caption{Description of Interrupt generated by the iridium module on an external digital input line.}
    \begin{tabular}{|m{0.25\textwidth}|m{0.75\textwidth}|}
    \hline
        \multicolumn{2}{|l|}{\textbf{Ring Indicator}} \\
        \hline
       \textbf{Entry Condition}  &  Buoy In any state other than reset with GPIO mapped to EXTI, with wake up from sleep mode. The WUP Pin receives a Digital High from Ring Indicator Pin on Iridium\\
       \hline
      \textbf{Function} &  The user has transmitted a packet to the buoy. Download the packet and execute/store the data based on the packet structure\\
       \hline
       \textbf{Exit Condition} & Device has downloaded user data which has been used to update the system and store data.\\
       \hline
       \textbf{Return State}& If entry source was a wake up, device will return to sleep. Otherwise device will return to the main loop.\\
       \hline
    \end{tabular}

    \label{tab:Int_desc_RI}
\end{table}

\begin{table}[H]
    \centering
    \caption{Description of routine for interrupts generated by the IMU on an external digital input line.}
    \begin{tabular}{|m{0.25\textwidth}|m{0.75\textwidth}|}
    \hline
        \multicolumn{2}{|l|}{\textbf{IMU Event Detection}} \\
        \hline
       \textbf{Entry Condition}  & Buoy In any state other than reset with GPIO wake up pin mapped to EXTI, with wake up from sleep mode. The WUP Pin receives a Digital High from Interrupt pin\\
       \hline
      \textbf{Function} & Device reads the interrupt source from the IMU, initializes I2C peripheral and begins sampling IMU data. Interrupt source determines the sampling rate, period and mode\\
       \hline
       \textbf{Exit Condition} & Device will exit when the IMU has finished sampling and the data has been stored into memory.\\
       \hline
       \textbf{Return State} & If entry source was a wake up, device will return to sleep. Otherwise device will return to the main loop.\\
       \hline
    \end{tabular}

    \label{tab:Int_desc_IMU}
\end{table}

\begin{table}[H]
    \centering
    \caption{Description of event handling routine for a brown out recovery event.}
    \begin{tabular}{|m{0.25\textwidth}|m{0.75\textwidth}|}
    \hline
    \multicolumn{2}{|l|}{\textbf{Brown out Detection}} \\
    \hline
    \textbf{Entry Condition}  & Buoy is in run mode or in Standby mode with Brown out detection voltage enabled. $V_{brownout}$ has been configured in option bytes. Event occurs when the voltage supplied to the microcontroller is less than $V_{brownout}$  causing the device to be held under reset. When the Voltage rises above the threshold, the device will enter the handler \\
    \hline
    \textbf{Function} & Device resets the relevant register flags and checks for data corruption. If no data is corrupted. Device will reload the last state and attempt to run it again. Otherwise the device performs a software reset \\
    \hline
    \textbf{Exit Condition} & $V_{supply} > V_{brownout}$, device successfully executes code in handler \\
    \hline
    \textbf{Return State} & Returns to main loop\\
    \hline
    \end{tabular}
    \label{tab:Ev_desc_BoD}
\end{table}

\begin{table}[H]
    \centering
    \caption{Description of routine for handling low power events.}
    \begin{tabular}{|m{0.25\textwidth}|m{0.75\textwidth}|}
    \hline
    \multicolumn{2}{|l|}{\textbf{Low Power Detection}} \\
    \hline
    \textbf{Entry Condition}  & Device is in run or sleep, Power Voltage thresholds set in PWR and interrupt enabled. Event occurs when $V_{supply} < V_{power}$ generating an event interrupt. \\
    \hline
    \textbf{Function} & Device will read INA sensor and transmit final packet to base. All peripherals switched off, Device placed into shut down mode.\\
    \hline
    \textbf{Exit Condition} & No Exit\\
    \hline
    \textbf{Return State} & No return state\\
    \hline
    \end{tabular}

    \label{tab:Ev_desc_LPD}
\end{table}

\begin{table}[H]
    \centering
    \caption{Description of routine for handling a software reset event.}
    \begin{tabular}{|m{0.25\textwidth}|m{0.75\textwidth}|}
    \hline
    \multicolumn{2}{|l|}{\textbf{Software Reset}} \\
    \hline
    \textbf{Entry Condition}  & The Microcontrollers NRST internal line is pulled low for a few seconds. This is triggered in any state by triggering a software reset in the NVIC\\
    \hline
    \textbf{Function} & Reset the buoy to an initial state. Clear any pending flags. Reset data in back up registers\\
    \hline
    \textbf{Exit Condition} & Successful reset of voltage domains\\
    \hline
    \textbf{Return State} & Return to Reset state and start of main loop\\
    \hline
    \end{tabular}

    \label{tab:Ev_desc_SWR}
\end{table}


\section{Initialization Routines}

\begin{table}[H]
    \centering
    \caption{Color guide for the initialization routine flow diagrams.}
    \begin{tabular}{l}
    \hline
       \cellcolor{micro}Microcontroller (HAL) function \\
        \cellcolor{sensor}Sensor (API) function \\
        \cellcolor{conditional}Function return statement evaluation \\
        \cellcolor{wrong}Fail return status \\
        \cellcolor{succ}Success return status\\
        \hline
    \end{tabular}

    \label{tab:Init_routine_Guide}
\end{table}


\begin{figure}[H]
    \centering
    \includegraphics[scale=0.3]{GPS Initialization Algorithm .png}
    \caption{Ublox Neo 7-m initialisation routine}
    \label{fig:Init_diagram_gps}
\end{figure}

\begin{figure}[H]
    \centering
    \includegraphics[scale=0.3]{Iridium Init Routine.png}
    \caption{Rockblock 9603 initialisation routine}
    \label{fig:Init_diagram_ir}
\end{figure}


\begin{figure}[H]
    \centering
    \includegraphics[scale=0.3]{Flash Chip Init Routine.png}
    \caption{AT45DB641E initialisation routine}
    \label{fig:Init_diagram_flash}
\end{figure}

\begin{figure}[H]
    \centering
    \includegraphics[scale=0.3]{Environmental Sensor Init Routine.png}
    \caption{BMP280 initialisation routine}
    \label{fig:Init_diagram_bmp}
\end{figure}

\begin{figure}[H]
    \centering
    \includegraphics[scale=0.3]{INA219 Init routine.png}
    \caption{INA219 initialisation routine}
    \label{fig:Init_diagram_ina}
\end{figure}

\begin{figure}[H]
    \centering
    \includegraphics[scale=0.3]{MPU Init Diagram.png}
    \caption{MPU6050 initialization routine}
    \label{fig:Init_diagram_mpu}
\end{figure}

\section{Code}

\subsection{BMP280 Temperature compensation formula}
\begin{figure}[H]

\begin{lstlisting}
/*
 * @brief Temperature Compensation algorithm
 *
 * @param T_val
 * @param t_fine
 * @param bmp_trim
 *
 * @retval int32_t
 */
int32_t BMP280_Compensate_Temp(int32_t T_val,int32_t* t_fine, BMP280_trim_t bmp_trim)
{
	//compensate Temperature from datasheet
	int32_t var1 = (((T_val>>3)- ((int32_t)bmp_trim.dig_T1<<1))*((int32_t)bmp_trim.dig_T2))>>11;
	int32_t var2 =  (((((T_val>>4) - ((int32_t)bmp_trim.dig_T1)) * ((T_val>>4) - ((int32_t)bmp_trim.dig_T1))) >> 12)*((int32_t)bmp_trim.dig_T3)) >> 14;
	int32_t temp = var1+var2; //for storage in global variable
	*t_fine = temp;
	return (temp*5 +128)/256;
}

\end{lstlisting}
    \caption{Function written to compensate  a 32 bit Temperature reading for sensor irregularities using the 32 bit version of the recommended compensation formula from the datasheet \cite{BMP280_Datasheet}. The formula uses the compensation parameters stored on the sensor}
    \label{fig:bmp_code_comp_T}
\end{figure}

\begin{figure}[H]
\begin{lstlisting}
/*
 * @brief Pressure compensation formula
 *
 * @param P_val
 * @param t_fine
 * @param bmp_trim
 *
 * @retval uint32_t
 */
uint32_t BMP280_Compensate_Pressure(uint32_t P_val,int32_t t_fine,BMP280_trim_t bmp_trim)
{
	//Compensation formula
	int32_t var1 = (int64_t)t_fine - 128000;
	int64_t var2 = var1*var1*((int64_t)(bmp_trim.dig_P6));
	var2 = var2 + (((int64_t)bmp_trim.dig_P4)<<35);
	var1 = ((var1 * var1 * (int64_t)bmp_trim.dig_P3)>>8) + ((var1 * (int64_t)bmp_trim.dig_P2)<<12);
	var1 = (((((int64_t)1)<<47)+var1))*((int64_t)bmp_trim.dig_P1)>>33;
	//check for divide by 0 error
	if(var1 == 0)return 0;
	int64_t P = 1048576 - (int32_t)P_val;
	P = (((P<<31)-var2)*3125)/var1;
	var1 = (((int64_t)(bmp_trim.dig_P9)) * (P>>13) * (P>>13)) >> 25;
	var2 = (((int64_t)(bmp_trim.dig_P8)) * P) >> 19;
	P = ((P + var1 + var2) >> 8) + (((int64_t)(bmp_trim.dig_P7))<<4);
	return (uint32_t)P;
}

\end{lstlisting}
    \caption{Function written to compensate  a 32 bit pressure reading for sensor irregularities using the 32 bit version of the recommended compensation formula from the datasheet \cite{BMP280_Datasheet}. The formula uses the compensation parameters stored on the sensor}
    \label{fig:bmp_code_comp_P}
\end{figure}

\subsection{INA219 Calibration Algorithm}
    \begin{lstlisting}[breaklines=true]
/*
 * Function Name INA_Status_t INA219_Calibrate_16V_1_2A(float *I_MBO, float *V_MBO, float *P_Max)
 * @brief: The following function writes a 16 bit value to the calibration register which
 * 			is used to adjust the current, bias voltage and power. Here, A LSB value is
 * 			calculated based on the user requirements and selected from a range. It would
 * 			be advisable to calculate the value manually and replace it in the function below
 * 			please note: the following function has values calculated manually. These can be
 * 			changed based on the configuration settings.
 * 			The values are calculated for 16V bus voltage range with a 2A expected current and
 * 			a 160mV shunt voltage range
 *
 * 	Step 1: V_Bus_Max = 16V
 * 			V_Shunt_Max = 160mV
 * 			R_Shunt = 0.1 Ohm
 *
 * 	Step 2: Max Possible I = 1.6A
 *
 * 	Step 3: Let I Max Expected = 1.2A
 *
 * 	Step 4: Min LSB = 36.6 uA/LSB
 * 			Max LSB = 292.97 uA
 *
 * 			Choose LSB = 100 uA
 * 	Step 5: Set Calibration value = 4096
 */

INA_Status_t INA219_Calibrate_16V_1_2A(float *I_MBO, float *V_MBO, float *P_Max)
{
	//set Current Step Size
	ina.INA219_I_LSB = 100.0/1000000.0;
	uint16_t I_cal_val = (uint16_t)(0.04096/(ina.INA219_I_LSB*INA219_R_SHUNT));
	ina.INA219_P_LSB = 20*ina.INA219_I_LSB;
	float I_max = ina.INA219_I_LSB*32767;
	if(I_max  > 1.6) //max possible current
	{
		*I_MBO = 1.6;
	}else
	{
		*I_MBO = I_max;
	}
	float Vshunt_max = *I_MBO*INA219_R_SHUNT;
	if(Vshunt_max > 0.16)
	{
		*V_MBO = 0.16;
	}
	else
	{
		*V_MBO = Vshunt_max;
	}
	*P_Max = *I_MBO*16;

	//write I_Cal_val to register
	uint8_t temp[2] = {(I_cal_val&0xFF00)>>8,(I_cal_val&0x00FF)};
	if(HAL_I2C_Mem_Write(&ina.ina_i2c,INA219_I2C_Address,CALIBRATION_REG,1,temp,2,100) != HAL_OK)
	{
		return INA_I2C_WRITE_ERROR;
	}

	return INA_OK;
}

\end{lstlisting}
\begin{center}
\captionsetup{type=figure}
\captionof{figure}{Calibration routine for INA219 Current sensor for a maximum current of 1.2A, maximum Bus Voltage of 16V and maximum shunt voltage of 160mV}
\label{fig:INA_Calib}
\end{center}

\subsection{Data Structs}
\begin{figure}[H]
    \centering
\begin{lstlisting}
/*
 * Coordinate Object
 *
 * Stores the Cordinates of GPS in the form DDMM.mmmm
 * where 
 *          DD     -   Degrees
 *          MM     -   Minutes
 *          mmmm   -   Fractional minutes
 * Variables:	Name.............Type.................................Description
 * 				lat..............float32_t............................GPS Lattitude
 * 				longi............float32_t............................GPS Longitude
 */
typedef struct
{
	float_t lat;
	float_t longi;
}Coord_t;
\end{lstlisting}    
    \caption{Coord\_t Data structure to store incoming GPS coordinates as IEEE754 32-bit floats}
    \label{fig:data_coord_t}
\end{figure}

\begin{figure}[H]
    \centering
\begin{lstlisting}
/*
 * Diagnostic Object
 *
 * Structure to Hold the GPS data signal diagnostics
 *
 * Variables:	Name.............Type.................................Description
 * 				PDOP.............DOP_t................................Positional Dilation of Precision (3D)
 * 				HDOP.............DOP_t................................Horizontal Dilation of Precision
 * 				VDOP.............DOP_t................................Vertical   Dilation of Precision
 * 				num_sats.........uint8_t..............................Number of Satelites used to obtain positional Fix
 * 				fix_type.........uint8_t..............................number between 1-3 describing the type of fix obtained
 *
 * Fix types
 * 1 - No Fix
 * 2 - 2D Fix (No altitude)
 * 3 - 3D Fix
 */

typedef struct
{
	DOP_t PDOP;
	DOP_t HDOP;
	DOP_t VDOP;
	uint8_t num_sats;
	uint8_t fix_type;
}Diagnostic_t;
\end{lstlisting}
    \caption{Data Structure for storing GPS signal diagnostic information}
    \label{fig:data_Diagnostic_t}
\end{figure}


\chapter{Supplementary Tables}

\begin{table}[H]
    \centering
    \caption{ List of data services provided by Iridium for transmission of data over the satellite network including the bandwidth and purpose of the service taken from \cite{iridium_mobile}}
    \label{tab:iridium service}
     \vspace{2.5mm}
    \begin{tabular}{|>{\RaggedRight}m{0.2\textwidth}|>{\RaggedRight}m{0.25\textwidth}| >{\RaggedRight}m{0.2\textwidth}| >{\RaggedRight}m{0.3\textwidth}|}
    \hline
    \textbf{Service Name} & \textbf{Purpose}& \textbf{Supporting Modems} & \textbf{Bandwidth}\\
    \hline
    Short Burst Data (SBD) & Sending short messages in bursts. & 
    \begin{itemize}
        \item 9603/9602
        \item Iridium Edge
        \item 9523
    \end{itemize}&
    \begin{itemize}
        \item 340 bytes upload \& 270 bytes download
        \item 1960 bytes upload \& 1890 bytes download
    \end{itemize} \\
    \hline
    Router-based Unrestricted Digital Interworking Connectivity Solution (RUDICS) & Continuous transfer of large real-time data from a large array of devices to a host. &
    \begin{itemize}
        \item 9523
        \item 9522B \par (\textit{depreciated})
    \end{itemize} &
    \begin{itemize}
        \item 6 – 10 Kbytes/min
    \end{itemize}\\
    \hline
    Circuit Switch Data (CSD) & Continuous transmission of large volumes of data over a dial-Up network using a SIM Card.& 
     \begin{itemize}
        \item 9523
        \item 9522B \par (\textit{depreciated})
    \end{itemize} &
    \begin{itemize}
        \item 6 – 10 Kbytes/min
    \end{itemize}\\
    \hline
    \end{tabular}
\end{table}

\newpage
\begin{center}
   { \setlength{\extrarowheight}{5pt}%
    \begin{longtable}[H]{|*{5}{>{\RaggedRight}m{0.18\textwidth}|}}
    \caption{Strategies used by the devices to transfer data from remote locations. Table includes transmission technologies and services used as well and transmission strategies and transmission intervals where given. Prices are converted to Rands (R) via \cite{usdcoversion}.}
    \label{tab:device_transmissionstrategies}\\
    \hline
      \textbf{Device Name}  & \textbf{Service} & \textbf{Modem} & \textbf{Bandwidth} & \textbf{Transmission Strategy}\\
       \hline
       WIIB  & Iridium SBD & 9602 & 340 bytes &  Data condensed into one 340 byte packet. (transmission intervals unavailable)\\
       \hline
       WIIOS & Iridium SBD & 9602 & 340 bytes & Data condensed into one 340 byte packet transmitted every 5 hours. \\
       \hline
       NDWB & Iridium RUDICS & 9522B & 6 - 10 Kybtes/min & raw inertial sample points transmitted every minute\\
        \hline
       SKIB & Iridium SBD (long range)\par ZigBee (short range)& \begin{itemize}
           \item 9602
           \item Xbee Pro
       \end{itemize}  & \begin{itemize}
           \item 340 bytes
           \item 50 Kbps
       \end{itemize} & \begin{itemize}
           \item GPS data transmitted every 10 minutes.
           \item Raw data transmitted when host is in range.
       \end{itemize}\\
        \hline
       SWIFT &  Iridium \par Ethernet & Iridium: \par 
         Geoforce SmartOne (tracking) \par 
         Unspecified SBD Modem (telemetry) \par
         Ethernet:\par 
         Digi Xpress ethernet bridge  & 
       Iridium: \par 
        N/A \par 
        1960 \par 
       Ethernet: \par 
       935 kb/s
     & Data transmitted through SBD modem. variable packet sizes ranging from 4 - 1228 bytes in length\\
        \hline
       SIMB & Iridium \par or ARGOS & 9603 & 340 & single packet transmission of 275 bytes \\
       \hline
       Polar ISVP & Iridium & 9602 & 340 bytes & User configured packet sizes and transmission intervals\\
       \hline
       Trident & Iridium & 9603 & 340 bytes & single packet transmission of 16 bytes \\
       \hline
    \end{longtable} }
\end{center}
\newpage

\chapter{Test Protocols}

\section{Unit Tests}
\label{app:Unittests}
\begin{table}[H]
	\centering
	\caption{Initialisation test outlining procedure, test cases and relation to
	 acceptance tests}
     \label{tab:UT001}
	\begin{tabular}{m{0.2\textwidth} m{0.7\textwidth}}
		\multicolumn{2}{l}{\textbf{UT001} }\\
		\hline
		\textbf{Description} & Peripheral initialization test\\
		\hline
		\hline
		\textbf{Test protocol} & The module is connected to the microcontroller with the corresponding subsystem API libraries and driver files. The initialization function is then run. After the function has completed, the return status from the function is evaluated. If the test was completed sucessfully, A "Module online!" is printed to the serial and the LED is switched on. \\
		\hline
		\hline
	\end{tabular}
\end{table}

\begin{table}[H]
	\centering
	\caption{Protocol for the deinitialisation test.}
	\label{tab:UT002}
	\begin{tabular}{m{0.2\textwidth} m{0.7\textwidth}}
		\multicolumn{2}{l}{\textbf{UT002} }\\
		\hline
		\textbf{Description} & Peripheral initialization test\\
		\hline
		\hline
		\textbf{Test protocol} & Test to be run after unit test UT001 in Table \ref{tab:UT001}. Module de initialisation function is to be run until completion. After the function has run, the configuration registers are checked for reset values. If the registers are reset, the function returns a success status and "Successfully deinitialised module!" is printed to the serial. If the LED was green, it is turned off. This test was also run before an initialisation function to ensure that the function does not result in hardfaults.\\
		\hline
		\hline
	\end{tabular}
\end{table}

\begin{table}[H]
	\centering
	\caption{Protocol for the transmission test.}
	\label{tab:UT003}
	\begin{tabular}{m{0.2\textwidth} m{0.7\textwidth}}
		\multicolumn{2}{l}{\textbf{UT003} }\\
		\hline
		\textbf{Description} & Transmission test\\
		\hline
		\hline
		\textbf{Test protocol} & Module to be connected to the microcontroller with firmware loaded with the required communication peripherals to be initialised. The microcontroller transmits a set of data to the device in polling mode. If succesful, the function will return a succesful return status. Otherwise the error code will be returned.  \\
		\hline
		\hline
	\end{tabular}
\end{table}

\begin{table}[H]
	\centering
	\caption{Reception Test}
	\label{tab:UT004}
	\begin{tabular}{m{0.2\textwidth} m{0.7\textwidth}}
		\multicolumn{2}{l}{\textbf{UT004} }\\
		\hline
		\textbf{Description} & Transmission test\\
		\hline
		\hline
		\textbf{Test protocol} & Module to be connected to the microcontroller with firmware loaded with the required communication peripherals to be initialised. The microcontroller transmits a request to recieve data in polling mode. The device polls the peripheral for the data for a user specified period of time. If succesful, the function will return a succesful return status. Otherwise the error code will be returned.    \\
		\hline
		\hline
	\end{tabular}
\end{table}

\begin{table}[H]
	\centering
	\caption{Reception Test}
	\label{tab:UT005}
	\begin{tabular}{m{0.2\textwidth} m{0.7\textwidth}}
		\multicolumn{2}{l}{\textbf{UT005} }\\
		\hline
		\textbf{Description} & Interrupt test\\
		\hline
		\hline
		\textbf{Test protocol} & Module to be connected to the microcontroller with firmware loaded with the required communication peripherals to be initialised. The required interrupt is configured and the interrupt condition is created to run after initialisation. An internal timer is set to a user specified value and is polled while the program runs. When the condition is met and the trigger is generated, the timer is stopped and the device should enter the interrupt handler. If this was successful, a "Pass" is printed to the serial. If a timeout occurs then a "Fail" is printed to the screen. \\
		\hline
		\hline
	\end{tabular}
\end{table}


\section{Subsystem test}
\begin{table}[H]
    \centering
    \caption{Subsystem test 2: Signal acquisition test outlining procedure, test cases and relation to acceptance tests.}
    \begin{tabular}{|m{0.2\textwidth}|m{0.75\textwidth}|}
    \hline
       \textbf{SUBSYS001: }  &  Wireless module signal acquisition test\\
       \hline
        \textbf{Input: } &  None\\
        \hline
        \textbf{Output: } & Serial output\\
        \hline
        \textbf{Tasks: } & \begin{enumerate}
        \vspace{1mm}
            \item Connect module to external power.
            \item place device in open environment free of obstructions.
            \item set timeout to 5 minutes.
            \item Request device return signal status
            \item evaluate return status
        \end{enumerate}\\
        \hline
        \textbf{Test Case: } & \begin{enumerate}
            \vspace{1mm}
            \item Open field, no obstructions - signal acquisition
            \item Open Field, Partial obstructions - slow signal acquisition
            \item Open Field, Full Obstructions - no signal acquisition
            \item Indoors, Partial Obstruction - slow signal acquisition
            \item Indoors, Full Obstructions - no signal acquisition
        \end{enumerate}\\
        \hline

    \end{tabular}
    \label{tab:SUBSYS001}
\end{table}


\begin{table}[H]
    \centering
    \caption{SUBSYS002: GPS data validity test outlining procedure, test cases and relation to acceptance tests}
    \begin{tabular}{|m{0.2\textwidth}|m{0.75\textwidth}|}
    \hline
       \textbf{SUBSYS002: }  &  GPS Data Validity Test\\
       \hline
        \textbf{Input: } &  NMEA Message String\\
        \hline
        \textbf{Output: } & Evaluation Result\\
        \hline
        \textbf{Tasks: } & \begin{enumerate}
        \vspace{1mm}
            \item Compare input packet structure to NMEA standard
            \item Compare packet address to accepted message strings and talker IDs
            \item Calculate checksum and compare to transmitted checksum byte
        \end{enumerate}\\
        \hline
        \textbf{Test Case: } & \begin{enumerate}
            \vspace{1mm}
            \item Valid GSA message string
            \item Valid GLL message string
            \item Valid ZDA message string
            \item Valid NMEA message with unrecognised address
            \item Valid NMEA message with invalid checksum
            \item Invalid message structure,
            \item Null string
        \end{enumerate}\\
        \hline
    \end{tabular}

    \label{tab:SUBSYS002}
\end{table}

\begin{table}[H]
    \centering
    \caption{SUBSYS003: DMA circular buffer input stream tests procedure and test case.}
    \begin{tabular}{|m{0.2\textwidth}|m{0.75\textwidth}|}
    \hline
       \textbf{SUBSYS003: }  &  DMA circular buffer verification test.\\
       \hline
        \textbf{Input: } &  Data of an unknown size\\
        \hline
        \textbf{Output: } & Return status and pointer to buffer in memory\\
        \hline
        \textbf{Tasks: } & \begin{enumerate}
        \vspace{1mm}
            \item Connect to device through communication port
            \item wait for data to be recieved
            \item validate return status
            \item validate data returned
            
        \end{enumerate}\\
        \hline
        \textbf{Test Case: } & \begin{enumerate}
            \vspace{1mm}
            \item byte
            \item array of known length
            \item string of unknown length
            \item No data 
            \item array of maximum integer size
        \end{enumerate}\\
        \hline
    \end{tabular}

    \label{tab:SUBSYS003}
\end{table}

\begin{table}[H]
    \centering
    \caption{SUBSYS004: Interrupt test proceedure,test cases and pass conditions}
    \begin{tabular}{|m{0.2\textwidth}|m{0.75\textwidth}|}
    \hline
       \textbf{SUBSYS004: }  &  Interrupt tests\\
       \hline
        \textbf{Input: } &  Interrupt request\\
        \hline
        \textbf{Output: } & Light emitting diode (LED) and information printed to serial monitor.\\
        \hline
        \textbf{Tasks: } & \begin{enumerate}
        \vspace{1mm}
            \item Configure interrupt in firmware to be enabled
            \item set the trigger condition
            \item Run the test and initiate the interrupt trigger condition
            \item If the code successfully enters the handler, flash an LED and print PASS to serial
        \end{enumerate}\\
        \hline
        \textbf{Test Case: } & \begin{enumerate}
            \vspace{1mm}
            \item Manual interrupt trigger
            \item Conditional interrupt trigger
            
        \end{enumerate}\\
        \hline
    \end{tabular}

    \label{tab:SUBSYS004}
\end{table}

\begin{table}[H]
    \centering
    \caption{SUBSYS005: Transmission test procedure, test cases and relation to acceptance tests}
    \begin{tabular}{|m{0.2\textwidth}|m{0.75\textwidth}|}
    \hline
       \textbf{SUBSYS005: }  &  Iridium transmission test\\
       \hline
        \textbf{Input: } &  message buffer\\
        \hline
        \textbf{Output: } & Transmission status\\
        \hline
        \textbf{Tasks: } & \begin{enumerate}
        \vspace{1mm}
            \item upload a message of known size to the module
            \item Initiate a Transmission
            \item evaluate return status
        \end{enumerate}\\
        \hline
        \textbf{Test Case: } & \begin{enumerate}
            \vspace{1mm}
            \item null message
            \item 1 byte message
            \item 50 byte message
            \item 120 byte message
            \item 340 byte message
            \item binary message
            \item ascii message
            
        \end{enumerate}\\
        \hline
    \end{tabular}

    \label{tab:SUBSYS005}
\end{table}

\begin{table}[H]
    \centering
    \caption{SUBSYS006: Temperature Verification test  test outlining procedure, test cases and relation to acceptance tests}
    \begin{tabular}{|m{0.2\textwidth}|m{0.75\textwidth}|}
    \hline
       \textbf{Unit Test 7: }  &  Environmental Sensor Temperature Validation Test\\
       \hline
        \textbf{Input: } &  Ambient Temperature\\
        \hline
        \textbf{Output: } &  Sensor Temperature Value\\
        \hline
        \textbf{Tasks: } & \begin{enumerate}
        \vspace{1mm}
            \item place sensor in an environment where the temperature is known
            \item Measure Ambient Temperature ADC value and read from sensor
            \item perform temperature compensation
            \item compare value to external measurement
        \end{enumerate}\\
        \hline
        \textbf{Test Case: } & \begin{enumerate}
            \vspace{1mm}
            \item $\pm 25\degree$ (Standard Temperature and Pressure)
            \item $10 \degree$ (Cold Test)
            \item $0\degree$ (Sub Zero Test)
            \item $-20 \degree$ (Extreme Freeze Test)
        \end{enumerate}\\
        \hline
    \end{tabular}

    \label{tab:SUBSYS006}
\end{table}

\begin{table}[H]
    \centering
    \caption{SUBSYS007: Inertial Measurement Unit test outlining procedure, test cases and relation to acceptance tests}
    \begin{tabular}{|m{0.2\textwidth}|m{0.75\textwidth}|}
    \hline
       \textbf{SUBSYS007: }  &  Inertial Measurement Unit Validation Test\\
       \hline
        \textbf{Input: } &  None\\
        \hline
        \textbf{Output: } & Test return status \\
        \hline
        \textbf{Tasks: } & \begin{enumerate}
        \vspace{1mm}
            \item perform Self-Test, verify self test values within $14\%$ of the factory set value
            \item enable all axes and read data
            \item enable interrupt pin and test response in Interrupt handler
            \item set device to sleep, wake up and take a reading
        \end{enumerate}\\
        \hline
        \textbf{Test Case: } & \begin{enumerate}
            \vspace{1mm}
                \item Device Connected 
                \item Device Disconnected
                \item Device at Rest
                \item Device in motion
        \end{enumerate}\\
        \hline
    \end{tabular}
    \label{tab:SUBSYS007}
\end{table}


\section{System test results}
\begin{table}[H]
    \centering
    \caption{Results of subsystem acceptance tests for each of the identified modules. Modules that were successfully validated were marked with a \checkmark, failed tests were marked by an X and tests that could not be applied to a subsystem were marked by an N/A}
    \begin{tabular}{|c|c|c|c|c|c|}
    \cline{2-6}
    \multicolumn{1}{c|}{}&\textbf{AT001}&\textbf{AT002}&\textbf{AT003}&\textbf{AT004 }&\textbf{AT005 }\\
    \hline
     GPS &  \checkmark & \checkmark & \checkmark & \checkmark & \checkmark\\
     \hline
     Iridium Modem &  \checkmark & \checkmark & \checkmark & \checkmark & \checkmark\\
     \hline
     Flash Chips &  \checkmark & \checkmark & \checkmark & \checkmark &
     \checkmark\\ 
     \hline
     Power Module & N/A & N/A & \checkmark & N/A& N/A\\
     \hline
     Env Sensor  &  \checkmark & \checkmark & \checkmark & \checkmark &
     \checkmark\\
     \hline
     Power Monitor   &  \checkmark & \checkmark & \checkmark & \checkmark &
     \checkmark\\
     \hline
     IMU   &  \checkmark & \checkmark & \checkmark & \checkmark &
     \checkmark\\
     \hline
    \end{tabular}

    \label{tab:AT_SSYS_EV}
\end{table}
